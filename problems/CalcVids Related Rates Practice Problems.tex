\documentclass[12pt]{report}
%\usepackage{amsmath,amsthm} %More Math Imputs
%\usepackage{amssymb} %More Math Symbols
%\usepackage{amsfonts} %More Math Fonts
\usepackage{times} %Times New Roman?
\usepackage{mathtools} %More Math Tools
\usepackage{changepage} %Changing Margins and Indent
%\usepackage{fancyhdr} %Fancy Header for making headers
\usepackage{extramarks} %Required for headers/footers?
\usepackage{enumerate} %Enumerate is half-assed....
\usepackage{graphicx} %Insert pictures
\usepackage{empheq} %Emphasize equations
\usepackage[none]{hyphenat}
\usepackage[titles]{tocloft}
\usepackage{multicol}
\usepackage[margin=1in]{geometry} %1inch margins?
\usepackage{multirow}
\pagenumbering{gobble}
\usepackage{fancyref}
\usepackage[explicit]{titlesec}
\usepackage{setspace}
%Margins
%\topmargin=-0.45in
%\evensidemargin=0in
%\oddsidemargin=0in
%\textwidth=6.5in %text area
%\textheight=9in %text area
%\headsep=.25in %separation between header and stuff
\linespread{1} %Line Spacing
\newcommand{\R}{\mathbb{R}}
\everymath{\displaystyle}
\usepackage{fancyhdr}
%\lhead{Zack Reed\\}
%\chead{RUME 2017 Proposal Tex}

%\lfoot{}
%\cfoot{\thepage}
%\rfoot{}
%\renewcommand\headrulewidth{0.4pt} %thickness of header line
%\renewcommand\footrulewidth{0.4pt} %thickness of footer line
\usepackage{amsfonts}
\usepackage{amsthm}
\usepackage{amsmath}
\usepackage{tabularx}
\setlength\parindent{.25in} % no indentation in the whole document
% Theorem, Definition, etc.... MACROS
\newtheorem{axiom}{Axiom}[subsection]
\newtheorem{definition}{Definition}[subsection]
\newtheorem{statement}{Statement}[subsection]
\newtheorem{comment}{Comment}[subsection]
\newtheorem{convention}{Convention}[subsection]
\newtheorem{proposition}{Proposition}[subsection]
\newtheorem{lemma}{Lemma}[subsection]
\newtheorem{theorem}{Theorem}[subsection]
\newtheorem{corollary}{Corollary}[subsection]
\newtheorem{properties}{Properties}[subsection]
\newtheorem{formula}{Formula}[subsection]
\newtheorem{example}{Example}[subsection]
%\newtheorem*{axiom}{Axiom}
\DeclareMathOperator{\Log}{Log}
\DeclareMathOperator{\Res}{Res}
\DeclareMathOperator{\Arg}{Arg}
\DeclareMathOperator{\Ind}{Ind}
\DeclareMathOperator{\Real}{Re}
\DeclareMathOperator{\Imag}{Im}
%\begin{definition}\textbf{}\\
%\end{definition}
%\begin{theorem}\textbf{}\\
%\end{theorem}
\newenvironment{myquote1}%
  {\list{}{\leftmargin=0.5in\rightmargin=.25in\linespread{1}}\item[]}%
  {\endlist}%Sets margins for blockquotes
 \newenvironment{myquote2}%
  {\list{}{\leftmargin=0.25in\rightmargin=0.25in}\item[]}%
  {\endlist}%Sets margins for blockquotes
  \usepackage{pdfpages}
\usepackage{graphicx}
%\setcounter{secnumdepth}{4}

\let\LaTeXStandardTableOfContents\tableofcontents

\renewcommand{\tableofcontents}{%
\begingroup%
\renewcommand{\bfseries}{\relax}%
\LaTeXStandardTableOfContents%
\endgroup%
}%
\usepackage[english]{babel}
\addto\captionsenglish{% Replace "english" with the language you use
  \renewcommand{\contentsname}%
    {\center{\normalsize TABLE OF CONTENTS}   \\  \hfill \normalsize\renewcommand{\bfseries}{\relax}\underline{Page}\vspace{-12pt}}%
    }

\setcounter{secnumdepth}{5}

\AtEndDocument{\thispagestyle{fancy}}
\AtEndDocument{ \fancyhead[R]{\thepage}}
\AtEndDocument{  \renewcommand{\headrulewidth}{0pt}}

\begin{document}
\newcommand{\epzero}{\forall\epsilon\textgreater0}
\newcommand{\epone}{\textless\epsilon}
\newcommand{\la}{\left|}
\newcommand{\ra}{\right|}
\newcommand{\lno}{\left\|}
\newcommand{\rno}{\right\|}
\newcommand{\xinf}{x\rightarrow\infty}
\newcommand{\less}{\textless}
\newcommand{\greater}{\textgreater}
\newcommand{\bff}{\begin{definition}}
\newcommand{\eff}{\end{definition}}
\newcommand{\reals}{\mathbb{R}}
\newcommand{\exN}{\exists N\in\mathbb{N}}
\newcommand{\tvect}[3]{%
\newcommand{\grad}{\nabla}
\newcommand{\weakarrow}{\hookrightarrow}
  \ensuremath{\Bigl(\negthinspace\begin{smallmatrix}#1\\#2\\#3\end{smallmatrix}\Bigr)}}
  \newcommand{\lnorm}{\left\|}
\newcommand{\rnorm}{\right\|}
\newcommand{\integers}{\mathbb{Z}}
\newcommand{\rationals}{\mathbb{Q}}
\newcommand{\Q}{\mathbb{Q}}
\newcommand{\F}{\mathbb{F}}
\newcommand{\jq}{\emph{Jerry}:}
\newcommand{\cq}{\emph{Christina}:}
\newcommand{\iq}{\emph{Interviewer}:}
\newcommand{\Lq}{\emph{Laura}:}
\newcommand{\N}{{\Bbb N}}
\newcommand{\Z}{{\Bbb Z}}
\newcommand{\dsp}{\displaystyle}
\newcommand{\blank}{\makebox[.5in]{\hrulefill}}
\newcommand{\be}{\begin{enumerate}}
\newcommand{\ee}{\end{enumerate}}
\newcommand{\limx}[1]{\lim_{x\rightarrow#1}}
\newcommand{\s}{\hskip6pt}

\section{Related Rates}

\begin{enumerate}

\item While sitting at an outdoor restaurant in a large city you notice that a hotel has an elevator on the outside of the building. You are 150 feet away from the hotel. 

\begin{center}
\includegraphics[width=3in]{Hotel_Math_S20.png}
\end{center}

What is the relationship between the rate of change of the height, $h$, of the elevator and the rate of change of the angle, $\theta$, between the ground and your line of sight?

\begin{enumerate}

\item $\frac{d\theta}{dt}=\left(h^2+150^2\right)\frac{dh}{dt}$

\item $\tan\frac{d\theta}{dt}=\frac{1}{150}\,\frac{dy}{dt}$

\item $\sec^2\theta\,\frac{d\theta}{dt}=\frac{1}{150}\,\frac{dy}{dt}
$

\item $\frac{d\theta}{dt}=\sqrt{150^2-2h}\frac{dh}{dt}$

\item $\frac{d\theta}{dt}=75h\frac{dh}{dt}$
\end{enumerate}

\pagebreak

\item A lighthouse is located on a small island 3 kilometers away from the nearest point $P$ on a straight shoreline. Let $x$ represent the distance between $P$ and the light beam's intersection with the shoreline. Also let $\theta$ represent the measure of the angle created by the beam of light and the line connecting the lighthouse and $P$. Which formula defines the relationship between the rate of change of the angle's measure and the rate at which the beam of light is moving along the shoreline?

\begin{center}
\includegraphics[width=5in]{Lighthouse_S20.png}
\end{center}

\begin{enumerate}

\item $\tan\left(\frac{d\theta}{dt}\right) = \frac{dx}{dt}$

\item $\sec^2(\theta)\cdot \frac{d\theta}{dt} = \frac{1}{3}\cdot\frac{dx}{dt}$

\item $\sec^2\left(\frac{d\theta}{dt}\right) = \frac{1}{3}\cdot\frac{dx}{dt}$

\item $\frac{d\theta}{dt} = \frac{1}{3\sec^2\left(\theta\right)}$

\item $\frac{d\theta}{dt} = \frac{1}{1+\left(\frac{dx}{dt}\right)^2}$

\end{enumerate}

\pagebreak


\item A cylindrical container of fixed radius $r$ is being filled with water. Which of the following equations expresses the relationship between the rate of change of the volume $V$ of the water in the container (with respect to time) and the rate of change of the height $h$ of the water in the container (with respect to time)?

\begin{enumerate}

\item $\frac{dV}{dt}=\pi r^2\cdot \frac{dh}{dt}$

\item $V=\pi r^2 h$

\item $\frac{dV}{dt}=2\pi rh+\pi r^2\cdot \frac{dh}{dt}$

\item $V=2\pi r h$

\item $\frac{dh}{dt}=\pi r^2h$

\end{enumerate}

\item A spherical ice ball of radius $r$ is melting in a liquid. It melts in a uniform fashion so that it remains a sphere while melting. Which of the following equations expresses the relationship between the rate of change of the volume $V$ of the ice (with respect to time) and the rate of change of its radius $r$ (with respect to time)?

\begin{enumerate}

\item $\frac{dV}{dt}=4\pi r^2\cdot \frac{dr}{dt}$

\item $\frac{dV}{dt}=4\pi r^2$

\item $V=\frac{4}{3}\pi r^3$

\item $\frac{dV}{dt}=4\pi\left(\frac{dr}{dt}\right)^2$

\item $\frac{dV}{dt}=\frac{4}{3}\pi r^3\cdot \frac{dr}{dt}$

\end{enumerate}

\item An ice cube that is initially three inches wide is placed on a table and starts to melt. (Assume that the ice cube melts in a uniform fashion so that at every instant it remains a cube.) Let $V$ denote the volume of the cube, measured in cubic inches, let $t$ denote the number of minutes elapsed since the cube began to melt, and let $x$ denote the width of the cube, measured in inches. If we know that the values of $x$ are related to the values of $t$ according to the formula $x=3e^{-t}$, then which of the following formulas correctly gives the instantaneous rate of change in $V$ with respect to $t$?

\begin{enumerate}

\item $V'(t)=-81e^{-3t}$

\item $V'(t)=\frac{27e^{-3t}-27}{t}$

\item $V'(t)=3e^{-t}$

\item $V'(t)=-3e^{-t}$

\item $V'(t)=9e^{-3t}$

\end{enumerate}

\item A stone thrown into a pond produces a circular ripple. If the radius of the ripple increases at a rate of 1.5 ft/sec, how fast is the area growing when the radius is 8 feet? State your answer as an exact value (i.e., in terms of $\pi$).

\item The hour hand of a clock tower is $5$ meters long. The minute hand is $7$ meters long. How fast, in meters/hr, is the distance between the tips of the hands changing at 3:00 AM?




\item Suppose gravel is being poured into a conical pile at a rate of 5 m$^3$/s, and suppose that the radius $r$ of this cone is always half its height $h$. How fast is the height of the pile increasing when the height is 10 m? \\ (\textit{Note that the formula for the volume of a cone is $\dsp V = \frac{1}{3}\pi r^2h$)}.\\
\includegraphics[width=2in]{Sandpile_F17.png}

\end{enumerate}



\end{document}