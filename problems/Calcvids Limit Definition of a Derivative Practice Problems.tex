\documentclass[12pt]{report}
%\usepackage{amsmath,amsthm} %More Math Imputs
%\usepackage{amssymb} %More Math Symbols
%\usepackage{amsfonts} %More Math Fonts
\usepackage{times} %Times New Roman?
\usepackage{mathtools} %More Math Tools
\usepackage{changepage} %Changing Margins and Indent
%\usepackage{fancyhdr} %Fancy Header for making headers
\usepackage{extramarks} %Required for headers/footers?
\usepackage{enumerate} %Enumerate is half-assed....
\usepackage{graphicx} %Insert pictures
\usepackage{empheq} %Emphasize equations
\usepackage[none]{hyphenat}
\usepackage[titles]{tocloft}
\usepackage{multicol}
\usepackage[margin=1in]{geometry} %1inch margins?
\usepackage{multirow}
\pagenumbering{gobble}
\usepackage{fancyref}
\usepackage[explicit]{titlesec}
\usepackage{setspace}
%Margins
%\topmargin=-0.45in
%\evensidemargin=0in
%\oddsidemargin=0in
%\textwidth=6.5in %text area
%\textheight=9in %text area
%\headsep=.25in %separation between header and stuff
\linespread{1} %Line Spacing
\newcommand{\R}{\mathbb{R}}
\everymath{\displaystyle}
\usepackage{fancyhdr}
%\lhead{Zack Reed\\}
%\chead{RUME 2017 Proposal Tex}

%\lfoot{}
%\cfoot{\thepage}
%\rfoot{}
%\renewcommand\headrulewidth{0.4pt} %thickness of header line
%\renewcommand\footrulewidth{0.4pt} %thickness of footer line
\usepackage{amsfonts}
\usepackage{amsthm}
\usepackage{amsmath}
\usepackage{tabularx}
\setlength\parindent{.25in} % no indentation in the whole document
% Theorem, Definition, etc.... MACROS
\newtheorem{axiom}{Axiom}[subsection]
\newtheorem{definition}{Definition}[subsection]
\newtheorem{statement}{Statement}[subsection]
\newtheorem{comment}{Comment}[subsection]
\newtheorem{convention}{Convention}[subsection]
\newtheorem{proposition}{Proposition}[subsection]
\newtheorem{lemma}{Lemma}[subsection]
\newtheorem{theorem}{Theorem}[subsection]
\newtheorem{corollary}{Corollary}[subsection]
\newtheorem{properties}{Properties}[subsection]
\newtheorem{formula}{Formula}[subsection]
\newtheorem{example}{Example}[subsection]
%\newtheorem*{axiom}{Axiom}
\DeclareMathOperator{\Log}{Log}
\DeclareMathOperator{\Res}{Res}
\DeclareMathOperator{\Arg}{Arg}
\DeclareMathOperator{\Ind}{Ind}
\DeclareMathOperator{\Real}{Re}
\DeclareMathOperator{\Imag}{Im}
%\begin{definition}\textbf{}\\
%\end{definition}
%\begin{theorem}\textbf{}\\
%\end{theorem}
\newenvironment{myquote1}%
  {\list{}{\leftmargin=0.5in\rightmargin=.25in\linespread{1}}\item[]}%
  {\endlist}%Sets margins for blockquotes
 \newenvironment{myquote2}%
  {\list{}{\leftmargin=0.25in\rightmargin=0.25in}\item[]}%
  {\endlist}%Sets margins for blockquotes
  \usepackage{pdfpages}
\usepackage{graphicx}
%\setcounter{secnumdepth}{4}

\let\LaTeXStandardTableOfContents\tableofcontents

\renewcommand{\tableofcontents}{%
\begingroup%
\renewcommand{\bfseries}{\relax}%
\LaTeXStandardTableOfContents%
\endgroup%
}%
\usepackage[english]{babel}
\addto\captionsenglish{% Replace "english" with the language you use
  \renewcommand{\contentsname}%
    {\center{\normalsize TABLE OF CONTENTS}   \\  \hfill \normalsize\renewcommand{\bfseries}{\relax}\underline{Page}\vspace{-12pt}}%
    }

\setcounter{secnumdepth}{5}

\AtEndDocument{\thispagestyle{fancy}}
\AtEndDocument{ \fancyhead[R]{\thepage}}
\AtEndDocument{  \renewcommand{\headrulewidth}{0pt}}

\begin{document}
\newcommand{\epzero}{\forall\epsilon\textgreater0}
\newcommand{\epone}{\textless\epsilon}
\newcommand{\la}{\left|}
\newcommand{\ra}{\right|}
\newcommand{\lno}{\left\|}
\newcommand{\rno}{\right\|}
\newcommand{\xinf}{x\rightarrow\infty}
\newcommand{\less}{\textless}
\newcommand{\greater}{\textgreater}
\newcommand{\bff}{\begin{definition}}
\newcommand{\eff}{\end{definition}}
\newcommand{\reals}{\mathbb{R}}
\newcommand{\exN}{\exists N\in\mathbb{N}}
\newcommand{\tvect}[3]{%
\newcommand{\grad}{\nabla}
\newcommand{\weakarrow}{\hookrightarrow}
  \ensuremath{\Bigl(\negthinspace\begin{smallmatrix}#1\\#2\\#3\end{smallmatrix}\Bigr)}}
  \newcommand{\lnorm}{\left\|}
\newcommand{\rnorm}{\right\|}
\newcommand{\integers}{\mathbb{Z}}
\newcommand{\rationals}{\mathbb{Q}}
\newcommand{\Q}{\mathbb{Q}}
\newcommand{\F}{\mathbb{F}}
\newcommand{\jq}{\emph{Jerry}:}
\newcommand{\cq}{\emph{Christina}:}
\newcommand{\iq}{\emph{Interviewer}:}
\newcommand{\Lq}{\emph{Laura}:}
\newcommand{\N}{{\Bbb N}}
\newcommand{\Z}{{\Bbb Z}}
\newcommand{\dsp}{\displaystyle}
\newcommand{\blank}{\makebox[.5in]{\hrulefill}}
\newcommand{\be}{\begin{enumerate}}
\newcommand{\ee}{\end{enumerate}}
\newcommand{\limx}[1]{\lim_{x\rightarrow#1}}
\newcommand{\s}{\hskip6pt}

\section*{Limit Definition of a Derivative}

\begin{enumerate}

\item Consider the function defined by $f(x) = x^2 - 1$. What is the value of $ \lim_{h \to 0}\frac{f(3+h)-f(3)}{(3+h)-3}$?

\begin{enumerate}

\item $0$

\item $6$

\item 8

\item $2x$

\item The limit does not exist.

\end{enumerate}

\item The expression $\lim_{h \to 0}\frac{(x+h)^3-\ln(x+h) - \left(x^3-\ln(x)\right)}{h}$ is the derivative of what function?

\begin{enumerate}

\item $f(x) = (x+h)^3-\ln(x+h)$

\item $f(x) = 3x^2 - \frac{1}{x}$

\item $f(x) = 3x^2 - \frac{1}{x}$

\item $f(x) = x^3 - \ln(x)$

\item $f(x) = \frac{(x+h)^3-\ln(x+h) - \left(x^3-\ln(x)\right)}{h}$

\end{enumerate}

\item What is the instantaneous rate at which the volume of an $8in^3$ cube grows as its side lengths increase from a single vertex on the left-most face?


\item If $f$ is a differentiable function and $a$ is a number, then $f^\prime(a)$ is given by which of the following expressions: \vspace{0.2cm}
\begin{enumerate}
\item [I.] $\dsp \lim_{h \to 0}\frac{f(a+h)-f(a)}{h}$ \vspace{0.2cm}
\item [II.] $\dsp \lim_{x \to a}\frac{f(x)-f(a)}{x-a}$ \vspace{0.2cm}
\item [III.] $\dsp \lim_{h \to 0}\frac{f(x+h)-f(x)}{x-h}$\\
\end{enumerate}
\begin{enumerate}
\item [a.] I only
\item [b.] II only
\item [c.] I and II only
\item [d.] I and III only
\item [e.] I, II, and III
\end{enumerate}

\item  The following expression represents the derivative of what function?\vspace{0.25cm}
\[
\lim_{\Delta x \to 0}\frac{2(x+\Delta x)^7-5(x+\Delta x)+8-\left(2x^7-5x+8\right)}{\Delta x}
\]\vspace{0.25cm}
\begin{enumerate}
\item [a.] $f(x) = 2(x+\Delta x)^7-5(x+\Delta x)+8$ \vspace{0.2cm}
\item [b.] $f(x) = 2x^7-5x+8$ \vspace{0.2cm}
\item [c.] $f(x) = 2(x+\Delta x)^7-5(x+\Delta x)+8-\left(2x^7-5x+8\right)$ \vspace{0.2cm}
\item [d.] $f(x) = 14x^6-5$ \vspace{0.2cm}
\item [e.] $\dsp f(x) = \frac{2(x+\Delta x)^7-5(x+\Delta x)+8-\left(2x^7-5x+8\right)}{\Delta x}$
\end{enumerate}

\item$\dsp \lim_{h \to 0} \frac{(2+h)^4-2^4}{h}=$ \vspace{0.25cm}
\begin{enumerate}
\item [a.] 0 \vspace{0.2cm}
\item [b.] 16 \vspace{0.2cm}
\item [c.] 1 \vspace{0.2cm}
\item [d.] 32 \vspace{0.2cm}
\item [e.] The limit does not exist
\end{enumerate}

\item The differentiable function $g$ is increasing over the interval $(x_0, \, x_0+1)$. If $x_0 < c < x_0+1$, what can you conclude about $\lim_{x \to c}\frac{g(x)-g(c)}{x-c}$? Explain your response. 

\end{enumerate}



\end{document}