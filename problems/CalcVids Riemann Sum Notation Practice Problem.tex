\documentclass[12pt]{report}

\usepackage{times} %Times New Roman?
\usepackage{mathtools} %More Math Tools
\usepackage{changepage} %Changing Margins and Indent
\usepackage{extramarks} %Required for headers/footers?
\usepackage{enumerate}
\usepackage{graphicx} %Insert pictures
\usepackage{empheq} %Emphasize equations
\usepackage[none]{hyphenat}
\usepackage[titles]{tocloft}
\usepackage{multicol}
\usepackage[margin=1in]{geometry} %1inch margins?
\usepackage{multirow}
\pagenumbering{gobble}
\usepackage{fancyref}
\usepackage[explicit]{titlesec}
\usepackage{setspace}
\linespread{1} %Line Spacing
\everymath{\displaystyle}
\usepackage{fancyhdr}
\usepackage{amsfonts}
\usepackage{amsthm}
\usepackage{amsmath}
\usepackage{tabularx}


\begin{document}
\newcommand{\less}{\textless}
\newcommand{\greater}{\textgreater}
\newcommand{\reals}{\mathbb{R}}
\newcommand{\integers}{\mathbb{Z}}
\newcommand{\rationals}{\mathbb{Q}}
\newcommand{\dsp}{\displaystyle}

\section*{Riemann Sum Notation}
\begin{enumerate}

\item Explain why $\sum_{k=1}^N f\left(3+\frac{(k-1)}{N}\right)\cdot\frac{1}{N}$ is an under approximation of the area bounded by the graph of $f$ and the $x$-axis when $f$ is increasing on the interval $[3, 4]$.

\item The function $y=g(t)$ represents the relationship between the rate of change in the value of investment stocks (in dollars per month) and the number of months $t$ elapsed since the stocks were purchased. Which of the following sums approximates the change in the value of the stocks over the interval of time from 4 to 7 months after the stocks were purchased?

\begin{enumerate}

\item $\sum_{k=4}^7g(k)$

\item $\sum_{k=4}^7g(t)\cdot\Delta t$

\item $\sum_{k=1}^6g(4+.5k)\cdot .5$

\item $\sum_{k=0}^3g(4+k)\cdot \Delta t$

\item $\sum_{k=0}^3g(4+k)$

\end{enumerate}



\item Let $f(x)$ represent the linear density (in g/m) of a 20 meter long wire, where $x$ is the distance in meters from one end. The mass of the wire is approximated by the left endpoint approximation with $N$ terms:

$$\sum_{i=1}^N f((i-1)\Delta x)\cdot \Delta x$$

Note that this Riemann sum is based on a uniform partition.

Explain what the following expressions represent {\bf in the context of the wire} and provide its units of measurement. 

\begin{enumerate}

\item $\Delta x$

\item $(i-1)\Delta x$

\item $f((i-1)\Delta x)$

\item $f((i-1)\Delta x)\cdot \Delta x$

\item $\sum_{k=1}^Nf((i-1)\Delta t)\cdot \Delta t$


\end{enumerate}


\item Let $f(t)$ represent the horizontal velocity (in ft/s) of a golf ball $t$ seconds after it was struck and lands $b$ seconds later. The horizontal distance traveled by the golf ball is approximated by the right endpoint approximation with $N$terms:

$$\sum_{i=1}^N f(i\Delta t)\cdot \Delta t$$

Note that this Riemann sum is based on a uniform partition.

Explain what the following expressions represent {\bf in the context of the golf ball} and provide its units of measurement. 

\begin{enumerate}

\item $\Delta t$

\item $i\Delta t$

\item $f(i\Delta t)$

\item $f(i\Delta t)\cdot \Delta t$

\item $\sum_{k=1}^Nf(i\Delta t)\cdot \Delta t$



\end{enumerate}

\item The function $g(t)$ gives the rate at which oil leaves a tanker, and is decreasing between $2$ minutes and $10$ minutes. Which of the following are underestimates of the amount of oil that left the tank between $2$ and $10$ minutes.

\begin{enumerate}

\item $g(2)+g(3)+g(4)+g(5)+g(6)+g(7)+g(8)+g(9)$

\item $\sum_{k=3}^{10}g(k)$

\item $\sum_{j=1}^{20}g\left(2+\frac{j-1}{2}\right)\cdot \frac{1}{2}$

\item $\sum_{j=1}^{20}g\left(2+\frac{j}{2}\right)\cdot \frac{1}{2}$

\item $2\cdot (g(4)+g(6)+g(8)+g(10))$

\item $2\cdot (g(3)+g(5)+g(7)+g(9))$



\end{enumerate}



\end{enumerate}

\end{document}